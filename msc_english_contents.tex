\chapter{Introduction}
\section{Problem statement and motivation}
The following gives some superficial instructions for using this template for a Master's thesis. For guidelines on thesis writing you can consult various sources, for example, the Bachelor thesis template.

The thesis should have an introduction chapter. Other chapters can be named according to the topic. In the end, some summary chapter is needed; see Chapter~\ref{chapter:conclusions} for an example.

\section{Structure of thesis}
Cool stuff etc.

\chapter{Neural networks}
Background stuff 
\section{Fully Connected Networks}
When needed, when good/bad
\section{Convolutional Neural Networks}
Show some basic CNN things maybe
\subsection{Neural image processing}
How it different from classical features and how it works
\section{Attention}
Attention huh
\section{Multi-task learning}
Describe how a multi-task setting differs
\section{RNN/LSTM/Transformer text processing/generation}
Depending on what may be used

\chapter{Tasks}
What has been done when solving these tasks/history/formal definition
\section{Image classification}
Cover some results on similar datasets and their approaches, maybe show that old SIFT/HOG features don't really work vs CNN
\subsection{Transfer learning}
\subsection{ResNet}
\subsection{EfficientNet}
\section{Object detection}
Based on what is going to be detected, maybe text?
\section{Paragraph captioning}
Assuming we are doing this
\section{Multi task learning}
Cover some examples of MTL in Computer Vision

\chapter{Datasets}
Describing our data
\section{Data formats}
Describe what datasets are used for each of the tasks and where they are from etc.

\chapter{Experiments}
Here are our experiments
\section{Training process}
\subsection{Evaluation criterias}
eval
\section{Image classifiers}
Results of image classsifiers on the datasets maybe compare ResNet vs EfficientNet
\subsection{Only transfer learned classifier}
Just a ResNet or something
\subsection{Image classifier with attention}
Add attention to image classifier
\subsection{Classifiers combined to a multi-task model}
How it do
\section{Object detection}
Some stuff for object detection
\section{Multi-task models}
Create multi-task models for all and try to find some pairs/tuples that actually work together properly
\chapter{Result analysis}
Incredible models absolutely.

\chapter{Future work}
How to make it better what could be tried 

\chapter{Figures and Tables}

\section{Figures}
Figure~\ref{fig:logo} gives an example how to add figures to the document. Remember always to cite the figure in the main text.

\begin{figure}[h!] 
\centering 
\includegraphics[width=0.3\textwidth]{HY-logo-ml.png}
\caption{University of Helsinki flame-logo for Faculty of Science.\label{fig:logo}}
\end{figure}

\section{Tables}

Table~\ref{table:results} gives an example how to report experimental results. Remember always to cite the table in the main text. 

\begin{table}
\centering
\caption{Experimental results.\label{table:results}}
\begin{tabular}{l||l c r} 
Experiment & 1 & 2 & 3 \\ 
\hline \hline 
$A$ & 2.5 & 4.7 & -11 \\
$B$ & 8.0 & -3.7 & 12.6 \\
$A+B$ & 10.5 & 1.0 & 1.6 \\
\hline
%
\end{tabular}
\end{table}

\chapter{Citations}

\section{Citations to literature}

References are listed in a separate .bib-file. In this case it is named \texttt{bibliography.bib} including the following content:
\begin{verbatim}
@article{einstein,
    author =       "Albert Einstein",
    title =        "{Zur Elektrodynamik bewegter K{\"o}rper}. ({German})
        [{On} the electrodynamics of moving bodies]",
    journal =      "Annalen der Physik",
    volume =       "322",
    number =       "10",
    pages =        "891--921",
    year =         "1905",
    DOI =          "http://dx.doi.org/10.1002/andp.19053221004"
}
 
@book{latexcompanion,
    author    = "Michel Goossens and Frank Mittelbach and Alexander Samarin",
    title     = "The \LaTeX\ Companion",
    year      = "1993",
    publisher = "Addison-Wesley",
    address   = "Reading, Massachusetts"
}
 
@misc{knuthwebsite,
    author    = "Donald Knuth",
    title     = "Knuth: Computers and Typesetting",
    url       = "http://www-cs-faculty.stanford.edu/%7Eknuth/abcde.html"
}
1
\end{verbatim}

In the last reference url field the code \verb+%7E+ will translate into \verb+~+ once clicked in the final pdf.

References are created using command \texttt{\textbackslash cite\{einstein\}}, showing as \citep{einstein}. Other examples: \citep{latexcompanion,knuthwebsite}.

Citation style can be negotiated with the supervisor. See some options in \url{https://www.sharelatex.com/learn/Bibtex_bibliography_styles}.

\section{Crossreferences}

Appendix~\ref{appendix:model} on page~\pageref{appendix:model} contains some additional material.

\chapter{From tex to pdf}

In Linux, run \texttt{pdflatex filename.tex} and \texttt{bibtex filename.tex} repeatedly until no more warnings are shown. This process can be automised using make-command.
 
\chapter{Conclusions\label{chapter:conclusions}}

It is good to conclude with a summary of findings. You can also use separate chapter for discussion and future work. These details you can negotiate with your supervisor.