\chapter{Introduction}
\section{Problem statement and motivation}
The following gives some superficial instructions for using this template for a Master's thesis. For guidelines on thesis writing you can consult various sources, for example, the Bachelor thesis template.

The thesis should have an introduction chapter. Other chapters can be named according to the topic. In the end, some summary chapter is needed; see Chapter~\ref{chapter:conclusions} for an example.

\section{Structure of thesis}
A lot of cool chapters here :))))))

\chapter{Neural networks}
Yes go through CNN, LSTM, Attention
\section{Convolutional Neural Networks}
CNN VERY COOL THING now if you need some help here I am
\subsection{Neural image processing}
How it different from classical and why it works
\subsection{Transfer learning}
How to transfer learning
\subsection{ResNet}
Describe them
\subsection{EfficientNet}
Describe them
\subsection{Image classification}
Tough problem ::)
\section{Image embedding}
pretty amazing
\section{Recurrent Neural Networks}
LSTMs extremely cool
\subsection{Recurrent networks over embeddings}
How dat works
\subsection{Video classification}
Video very cool
\section{Attention}
Attention huh
\section{Multi-task learning}
What it do
\chapter{Related Work}
How did the others do and me also
\section{Non-neural}
Some cool old school pixel approaches
\section{CNN / Pretrained CNN}
How did they train their model
\section{Video}
How to use video\cite{guerra_weather_2018}
\chapter{Dataset}
Describing our data
\section{Data format}
\section{Data processing and labeling}
\section{Semi-supervised learning}


\chapter{Experiments}
Here are our experiments
\section{Training process}
\subsection{Evaluation criteria}
eval
\subsection{Mixed precision training}
16 bit  masters
\section{Image classifiers}
Only use images
\subsection{Only transfer learned classifier}
Just a ResNet or something
\subsection{Image classifier with attention}
Add attention to image classifier
\section{Video classification}
\subsection{LSTM over video}
just lstm
\subsection{Attention over video}
Attention over video here

\chapter{Result analysis}
Very good models

\chapter{Future work}
How to make it better

\chapter{Figures and Tables}

\section{Figures}
Figure~\ref{fig:logo} gives an example how to add figures to the document. Remember always to cite the figure in the main text.

\begin{figure}[h!] 
\centering 
\includegraphics[width=0.3\textwidth]{HY-logo-ml.png}
\caption{University of Helsinki flame-logo for Faculty of Science.\label{fig:logo}}
\end{figure}

\section{Tables}

Table~\ref{table:results} gives an example how to report experimental results. Remember always to cite the table in the main text. 

\begin{table}
\centering
\caption{Experimental results.\label{table:results}}
\begin{tabular}{l||l c r} 
Experiment & 1 & 2 & 3 \\ 
\hline \hline 
$A$ & 2.5 & 4.7 & -11 \\
$B$ & 8.0 & -3.7 & 12.6 \\
$A+B$ & 10.5 & 1.0 & 1.6 \\
\hline
%
\end{tabular}
\end{table}

\chapter{Citations}

\section{Citations to literature}

References are listed in a separate .bib-file. In this case it is named \texttt{bibliography.bib} including the following content:
\begin{verbatim}
@article{einstein,
    author =       "Albert Einstein",
    title =        "{Zur Elektrodynamik bewegter K{\"o}rper}. ({German})
        [{On} the electrodynamics of moving bodies]",
    journal =      "Annalen der Physik",
    volume =       "322",
    number =       "10",
    pages =        "891--921",
    year =         "1905",
    DOI =          "http://dx.doi.org/10.1002/andp.19053221004"
}
 
@book{latexcompanion,
    author    = "Michel Goossens and Frank Mittelbach and Alexander Samarin",
    title     = "The \LaTeX\ Companion",
    year      = "1993",
    publisher = "Addison-Wesley",
    address   = "Reading, Massachusetts"
}
 
@misc{knuthwebsite,
    author    = "Donald Knuth",
    title     = "Knuth: Computers and Typesetting",
    url       = "http://www-cs-faculty.stanford.edu/%7Eknuth/abcde.html"
}
\end{verbatim}

In the last reference url field the code \verb+%7E+ will translate into \verb+~+ once clicked in the final pdf.

References are created using command \texttt{\textbackslash cite\{einstein\}}, showing as \citep{einstein}. Other examples: \citep{latexcompanion,knuthwebsite}.

Citation style can be negotiated with the supervisor. See some options in \url{https://www.sharelatex.com/learn/Bibtex_bibliography_styles}.

\section{Crossreferences}

Appendix~\ref{appendix:model} on page~\pageref{appendix:model} contains some additional material.

\chapter{From tex to pdf}

In Linux, run \texttt{pdflatex filename.tex} and \texttt{bibtex filename.tex} repeatedly until no more warnings are shown. This process can be automised using make-command.
 
\chapter{Conclusions\label{chapter:conclusions}}

It is good to conclude with a summary of findings. You can also use separate chapter for discussion and future work. These details you can negotiate with your supervisor.