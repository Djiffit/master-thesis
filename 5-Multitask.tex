\chapter{Multi-task learning}
Maybe drop some sections if it becomes too long.
\section{Definition}
Formally define multi-task learning similar to Transfer Learning using the original multi-task paper \citep{origMultitask}, and a newer view on the domain \citep{surveyOnMultiTask}.
\section{Benefits}
Inference time, regularization, smaller models etc. Weighing different tasks and getting information from the loss functions \citep{lossWeighting} \citep{usingUncertaintyToWeighLosses}
\section{Soft parameter sharing}
Define soft parameter sharing between tasks, show an example from \citep{mutualLearning}.
\section{Hard parameter sharing}
Show how multiple heads can be combined to produce outputs for various tasks using different heads, show results from sharing all/some layers. \citep{visualPerson} \citep{selfDriving} \citep{healthyDrink}
\section{Special multi-task learning techniques}
Show how an attention layer can be added to augment the performance in various tasks in a multi-task setting \citep{multiTaskAttention}, specifically when the tasks are correlated \citep{multiTaskWeather} \citep{weatherNet}. Also cross-stich networks \citep{crossStitch}.
\section{Multi-task learning with little data}
Why it can be beneficial with little data \citep{surveyOnMultiTask} Show some results where it is beneficial on little data.
\section{What and when to share}
Some ways to decide when multi-task learning might be a good idea and how to evaluate which tasks might be good to learn in a multi-task setting \citep{taskonomy} \citep{whichTasks}.
Show how this problem can be similar to deciding whether Transfer Learning is a good idea \citep{whatAndWhereToTransfer} \citep{transferringMidLevelRepresentations}
Also cover the difficulties when having to share things.