\chapter{Object detection}
Object detection is another prevalent task in the domain of computer vision.
An object detection task is one where the goal is to localize one or many different classes of objects using bounding boxes.
Before current deep learning-based techniques, a popular way to solve the problem was to use handcrafted features like in image classification and to use a sliding window over the image for localizing objects.
These previous techniques include, for example, the Viola-Jones detector \citep{viola-jones}, which uses a sliding window and AdaBoost for features, and another popular method was using histograms of oriented gradients \citep{hogs} to find where the boundaries of objects exist.
These days there exist various ways of using the feature maps of neural networks for learning the filters to get much better results.
The various architectures can be split into two approaches, one-stage detectors, and two-stage detectors.
The two-stage detectors require region proposals, based on which the object detection is done.
An often-used family of these kinds of methods is the R-CNN classifiers, for example, the faster R-CNN \citep{faster-rcnn}.
In the one-stage detectors, features from various layers of the classifier are used to get the predictions.
For example, the YOLOv4 \citep{yolov4} is the 4th iteration of the single-stage YOLO family of detectors that are very popular due to the good balance between fast speed and good accuracy.