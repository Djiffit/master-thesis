%% History:
%% May 2019 Tomi Männistö, Antti-Pekka Tuovinen proofreading; 30 vs. 40 cr theses, etc.
%% May 2019 Tomi Männistö changed from babelbib to bibtex; Abstract page (and other pages as well) reformatting.
%% January–May 2019 several issues fixed by Niko Mäkitalo; long fields in abstract
%% March 2018 template file extended by Lea Kutvonen to exploit HYthesisML.cls.
%% Feb2018 This template file for the use of HYgraduML.cls was  modified by Veli Mäkinen from HY_fysiikka_LuKtemplate.tex
%% authored by Roope Halonen ja Tomi Vainio in 2017.
%% Some text is also inherited from engl_malli.tex versions by Kutvonen, Erkiö, Mäkelä, Verkamo, Kurhila, and
%% Nykänen, to accompany tktltiki.cls (by Puolakka 2002).


%% Follow comments to support use.

%%%%%%%%%%%%%%%%%%%%%%%%%%%%%%%%%%%%%%%%%%%%%%%%%%%%%%%%%
%% STEP 1: Choose options for MSc / BSc layout and your bibliographic style
%%%%%%%%%%%%%%%%%%%%%%%%%%%%%%%%%%%%%%%%%%%%%%%%%%%%%%%%%

%%  Language: 
%%      finnish, swedish, or english
%%  Pagination (use twoside by default)  
%%      oneside or twoside,
%%  Study programme / kind of report
%%      csm  = pro gradu in new Computer science MSc;
%%      cs = pro gradu in old Computer Science MSc;
%%      tkt = BSc thesis in new curricula;
%%      tktl= BSc thesis in old curricula;
%%  For MSc choose your line or track:
%%      (30 cr thesis, 2020 onwards, Master of Computer Science programme = csm)
%%      software-track-2020 = Software study track
%%      algorithms-track-2020 = Algorithms study track
%%      networking-track-2020 = Networking study track
%%
%%      (30 cr thesis, Master of Computer Science programme = csm)
%%      sw-track-2018 = Software Systems study track
%%      alko-track-2018 = Algorithms study track
%%      nodes-track-2018 = Networking and Services study track
%%
%%      (30 cr thesis, Master of Computer Science programme = csm)
%%      sw-line-2017 =  Software systems subprogramme
%%      alko-line-2017 = Algorithms, Data Analytics and Machine Learning subprogramme
%%      bio-line-2017 = Algorithmic Bioinformatics subprogramme
%%      nodes-line-2017 = Networking and Services subprogramme
%%
%%      (40 cr thesis, = cs)
%%      sw-line = Software Systems specialisation line 
%%      alko-line = Algorithms specialisation line
%%      bio-line = Algorithmic bioinformatics specialisation line
%%      nodes-line = Networking and Services specialisation line

\documentclass[english,twoside,censored,csm,algorithms-track-2020]{HYthesisML}


% In theses, open new chapters only at right page.
% For other types of documents, may ask "openany" in document.
\PassOptionsToClass{openright,twoside,a4paper}{report}
%\PassOptionsToClass{openany,twoside,a4paper}{report}

\usepackage{csquotes}
%%%%%%%%%%%%%%%%%%%%%%%%%%%%%%%%%%%%%%%%%%%%%%%%%%%%%%%%%
%% REFERENCES
%% Some notes on bibliography usage and options:
%% natbib -> you can use, e.g., \citep{} or \parencite{} for (Einstein, 1905); with APA \cite -> Einstein, 1905 without ()
%% maxcitenames=2 -> only 2 author names in text citations, if more -> et al. is used
%% maxbibnames=99 as no great need to suppress the biliography list in a thesis
%% for more information see biblatex package documentation, e.g., from https://ctan.org/pkg/biblatex 

%% Reference style: select one 
%% for APA = Harvard style = authoryear -> (Einstein, 1905) use:
\usepackage[style=authoryear,bibstyle=authoryear,backend=biber,natbib=true,maxnames=99,maxcitenames=2,giveninits=true,uniquename=init]{biblatex}
%% for numeric = Vancouver style -> [1] use:
%\usepackage[style=numeric,bibstyle=numeric,backend=biber,natbib=true,maxbibnames=99,giveninits=true,uniquename=init]{biblatex}
%% for alpahbetic -> [Ein05] use:
%\usepackage[style=alphabetic,bibstyle=alphabetic,backend=biber,natbib=true,maxbibnames=99,giveninits=true,uniquename=init]{biblatex}
%

\addbibresource{bibliography.bib}
% in case you want the final delimiter between authors & -> (Einstein & Zweistein, 1905) 
% \renewcommand{\finalnamedelim}{ \& }
% List the authors in the Bibilipgraphy as Lastname F, Familyname G,
\DeclareNameAlias{sortname}{family-given}
% remove the punctuation between author names in Bibliography 
%\renewcommand{\revsdnamepunct}{ }


%% Block of definitions for fonts and packages for picture management.
%% In some systems, the figure packages may not be happy together.
%% Choose the ones you need.

%\usepackage[utf8]{inputenc} % For UTF8 support, in some systems. Use UTF8 when saving your file.

\usepackage{lmodern}         % Font package, again in some systems.
\usepackage{textcomp}        % Package for special symbols
\usepackage[pdftex]{color, graphicx} % For pdf output and jpg/png graphics
\usepackage{epsfig}
\usepackage{subfigure}
\usepackage{algorithm}
\usepackage[algo2e]{algorithm2e}
\usepackage[]{algpseudocode}
\usepackage[pdftex, plainpages=false]{hyperref} % For hyperlinks and pdf metadata
\usepackage{fancyhdr}        % For nicer page headers
\usepackage{tikz}            % For making vector graphics (hard to learn but powerful)
%\usepackage{wrapfig}        % For nice text-wrapping figures (use at own discretion)
\usepackage{amsmath, amssymb} % For better math

\singlespacing               %line spacing options; normally use single

\fussy
%\sloppy                      % sloppy and fussy commands can be used to avoid overlong text lines
% if you want to see which lines are too long or have too little stuff, comment out the following lines
% \overfullrule=1mm
% to see more info in the detailed log about under/overfull boxes...
% \showboxbreadth=50 
% \showboxdepth=50



%%%%%%%%%%%%%%%%%%%%%%%%%%%%%%%%%%%%%%%%%%%%%%%%%%%%%%%%%
%% STEP 2:
%%%%%%%%%%%%%%%%%%%%%%%%%%%%%%%%%%%%%%%%%%%%%%%%%%%%%%%%%
%% Set up personal information for the title page and the abstract form.
%% Replace parameters with your information.
\title{OwO}

% TM: Contributors to template editors now listed in the beginning of the file in comments
\author{Konsta Kutvonen}
\date{\today}



% Set supervisors and examiners, use the titles according to the thesis language
% Prof. 
% Dr. or in Finnish toht. or tri or FT, TkT, Ph.D. or in Swedish... 
\supervisors{Prof.~D.U.~Mind}
\examiners{Prof.~D.U.~Mind, Dr.~O.~Why}


\keywords{ulkoasu, tiivistelmä, lähdeluettelo}
\additionalinformation{\translate{\track}}

%% For seminar papers and such, cover page and abstract
%% requires these three basic items of information.
%% Label as needed and remove comment marks.
%%\level{Seminaariraportti}
%%\programme{Tietojenkäsittelytieteen andidaattiohjelma}
%%\subject{Seminaarisarjan nimi}

%% Replace classification terms with the ones that match your work. ACM
%% ACM Digital library provides a taxonomy and a tool for classification
%% in computer science. Use 1-3 paths, and use right arrows between the
%% about three levels in the path; each path requires a new line.

\classification{\protect{\ \\
\  General and reference $\rightarrow$ Document types  $\rightarrow$ Surveys and overviews\  \\
\  Applied computing  $\rightarrow$ Document management and text processing  $\rightarrow$ Document management $\rightarrow$ Text editing
}}

%% if you want to quote someone special. You can comment this line out and there will be nothing on the document.
%\quoting{Bachelor's degrees make pretty good placemats if you get them laminated.}{Jeph Jacques}


%% OPTIONAL STEP: Set up properties and metadata for the pdf file that pdfLaTeX makes.
%% Your name, work title, and keywords are recommended.
\hypersetup{
    unicode=true,           % to show non-Latin characters in Acrobat’s bookmarks
    pdftoolbar=true,        % show Acrobat’s toolbar?
    pdfmenubar=true,        % show Acrobat’s menu?
    pdffitwindow=false,     % window fit to page when opened
    pdfstartview={FitH},    % fits the width of the page to the window
    pdftitle={},            % title
    pdfauthor={},           % author
    pdfsubject={},          % subject of the document
    pdfcreator={},          % creator of the document
    pdfproducer={pdfLaTeX}, % producer of the document
    pdfkeywords={something} {something else}, % list of keywords for
    pdfnewwindow=true,      % links in new window
    colorlinks=true,        % false: boxed links; true: colored links
    linkcolor=black,        % color of internal links
    citecolor=black,        % color of links to bibliography
    filecolor=magenta,      % color of file links
    urlcolor=cyan           % color of external links
}

%%-----------------------------------------------------------------------------------

\begin{document}

% Generate title page.
\maketitle


%%%%%%%%%%%%%%%%%%%%%%%%%%%%%%%%%%%%%%%%%%%%%%%%%%%%%%%%%
%% STEP 3:
%%%%%%%%%%%%%%%%%%%%%%%%%%%%%%%%%%%%%%%%%%%%%%%%%%%%%%%%%
%% Write your abstract to be positioned here.
%% You can make several abstract pages (if you want it in different languages),
%% but you should also then redefine some of the above parameters in the proper
%% language as well, in between the abstract definitions.

\begin{otherlanguage}{english}
    \begin{abstract}
        Use this otherlanguage environment to write your abstract in another language if needed.

        Topics are classified according to the ACM Computing Classification System
        (CCS), see
        \url{https://www.acm.org/about-acm/class}:
        check command \verb+\classification{}+. A small set of paths (1--3) should be used, starting from any top nodes
        referred to bu the root term CCS leading to the leaf nodes. The elements
        in the path are separated by right arrow, and emphasis of each element individually can be indicated
        by the use of bold face for high importance or italics for intermediate
        level. The combination of individual boldface terms may give the reader
        additional insight.
    \end{abstract}
\end{otherlanguage}

% Place ToC
\newpage
\mytableofcontents
\frontmatter

%%%%%%%%%%%%%%%%%%%%%%%%%%%%%%%%%%%%%%%%%%%%%%%%%%%%%%%%%
%% STEP 4: Write the thesis.
%%%%%%%%%%%%%%%%%%%%%%%%%%%%%%%%%%%%%%%%%%%%%%%%%%%%%%%%%
%% Your actual text starts here. You shouldn't mess with the code above the line except
%% to change the parameters. Removing the abstract and ToC commands will mess up stuff.
%%
%% You may wish to include material to avoid browsing the definitions
%% above. Command \include{file} includes the file of name file.tex.
%% As a side effect, subsequent inclusions may force a page break.

% BSc instructions
%\include{bsc_finnish_contents}
%\include{bsc_english_contents}
% MSc instructions
%\include{msc_finnish_contents}
\chapter{Introduction}
Since 2012 computer vision research has been increasingly dominated by approaches using deep Convolutional Neural Networks (CNNs). These CNN based techniques have allowed for achieving significantly better results in computational image and video understanding compared to previous state-of-the-art approaches. Nowadays, there are many interesting avenues of computer vision research, such as image segmentation, object detection, object recognition, image captioning, pose detection, and many more. Since the task of computational image understanding is quite complex, the CNNs used to solve these problems often require dozens, if not hundreds of millions of parameters, in order to achieve sufficiently reliable accuracies. As the number of parameters is high, the networks require large computational resources to find optimal values for all of the variables in all of the hundreds of layers of matrices. Graphics Processing Units (GPUs) or Tensor Processing Units (TPUs) are the tools that people use to train the parameters in the networks since the training operations are highly parallelizable, allowing for the training of networks with hundreds or even thousands of layers. The problem often is that when training, one wants to increase the batch size, but at the same time, the computations must fit in the memory of the processing unit, which can be difficult to increase. Also, as the number of parameters in the model increases, the memory required for training the networks, and the time to produce outputs increases. To combat these problems, we often have to get better or more hardware, which can be expensive or compromise in the size of the model, which may lead to unreliable performance.

Embedded devices can use these CNN based algorithms to analyze their surroundings in an environment where they can't rely on external predictions from the cloud. One of the most significant issues is that achieving scene understanding requires multiple classifiers to solve various tasks. For example, a camera-based self-driving car has numerous things it is interested in resolving using the video output of its cameras. The tasks could be, for example, finding and classifying other vehicles, traffic signs, road markings, predicting paths of vehicles, pedestrians, and other actors, understanding where it can move and finally producing the output of steering angle and acceleration. All these tasks have to be solved using the limited computing capability onboard the actor, and the inference time for them can't be too long in order to react to everything within a reasonable time frame. Here we have many tasks, and if every task requires its independent network to produce the outputs, and we cut down on the network sizes, the classifiers might be powerful enough to produce safe outputs. On the other hand, if we don't reduce the network sizes, our inference time might be too long for a real-time system. One way to possibly get a good compromise between inference time and model complexity is to use multi-task learning and weight sharing within the model to get a single model with multiple outputs and similar or better performance to individual classifiers.

Multi-task learning proposes some solutions to these problems but makes the training process more difficult. Still, in some cases, it is something that has to be done to get a good enough inference time and accuracy on the problem at hand. Many modern computer vision models already feature an ImageNet backbone as a part of the classifier, so this is a good part of the network to consider for sharing, but finding which tasks and how much can be shared is a difficult problem to solve.


\chapter{Neural networks}
Basic background stuff, maybe cover little if there is room otherwise just add a reference to some basic material
\section{Fully Connected Networks}
When needed example of a transfer learning head?, when good/bad
\section{Convolutional Neural Networks}
Show some basic CNN things maybe and why needed over fully connected bois :)
\chapter{Image classification}
Image classification is one of the essential modern computer vision problems, where the goal is to create a model that can classify an input image into one of a set of pre-defined classes. Before the popularization of applying large Convolutional Neural Networks for this task, the most successful way of solving the problem was to use some algorithm for finding feature descriptors in a set of images to construct a Visual Bag of Words.  A linear classifier, like a Support Vector Machine, would then do the classification using the Bag of Words representation of an image. These days nearly all approaches are based on using deep CNNs, and working CNNs were deployed already in 1998 on character recognition in the form of LeNet \citep{leNet}.

\section{ImageNet}

ImageNet  \citep{imagenet}  is perhaps the most significant dataset for image classification and especially the ImageNet Challenge \citep{ILSVRC}, which is a challenge for a collection of 1000 classes from the ImageNet dataset for image classification using 1.2 million training and 150 thousand test images of the entire ImageNet dataset. In 2012 the winning model, AlexNet \citep{alexNet}, showed that it was possible to train a deep CNNs efficiently using GPUs. Since 2012 all top-performing models showed some new improvements on how to create the most performant network architecture, for example, VGGNet from the year 2014 and ResNet \citep{resNet} from 2015, both of which have been popular models to use for Transfer Learning since.

Human accuracy on the ImageNet challenge is about 5.1\% \citep{imageNet_summary}, and ResNet achieves a top-5 error rate of 3.57\% \citep{resNet} and newer architectures even lower, but this still does not mean that image classification is a solved problem. The human performance experiment found that many of the human errors are caused by not having expert information in, for example, identifying animal species or not even being aware of the existence of a class \citep{imageNet_summary}. ObjectNet \citep{objectNet} is a dataset designed to test image classifiers with a focus on their generalizability. It contains many classes that also exist in the ImageNet dataset. However, they are in unexpected locations or have an unexpected pose, causing the high accuracy image classifiers trained on ImageNet to experience a 40-45\% accuracy drop when evaluating them on the ObjectNet images of classes shared between ImageNet and ObjectNet. This kind of adjustment is relatively easy for a human, and it shows that while the classifiers are good, they are by no means perfect.

\section{Transfer learning}
Transfer learning is a powerful technique to obtain results quickly when using deep CNNs. Here we will only take a look at transfer learning within the domain of deep CNNs, but it is a technique that has been successfully applied to many other domains of machine learning as well.

To give a formal definition of transfer learning, we will follow the definitions provided in \citep{transferSurvey2010}. Let $\mathcal{D}$ be a domain, which consists of a feature space $\mathcal{X}$ and a marginal probability distribution P($\mathcal{X}$). For a given domain, a task $\mathcal{T}$ = {$\mathcal{Y}$, f$\mathcal(X)$} consist of a label space $\mathcal{Y}$ for the inputs and of a predictive function $f(x)$ which produces predictions for all pairs ${x_i, y_i}$ where $x_i \in \mathcal{X}$ and $y_i \in \mathcal{Y}$. Given a source domain $\mathcal{D}_S$, a source task $\mathcal{T}_S$, a target domain $\mathcal{D}_T$ and a target task $\mathcal{T}_T$, where target and source are disjoint, transfer learning tries to improve the performance of $f_T(x)$ using $\mathcal{D}_S$ and $\mathcal{T}_S$.

Unlike the ImageNet challenge, most real-world tasks do not have such an abundance of data for all possible classes. Still, to achieve the highest accuracies, they require models that are equal in terms of complexity to those that have top accuracies problems on the scale of the ImageNet classification. For this reason, many CNN classifiers, irrespective of the problem, feature one of the ImageNet classifiers in the model architecture. Even though some datasets may contain a large number of images per class, using a pre-trained classifier as a basis often produces a better final classifier by applying fine-tuning \citep{betterTransfer}.

The idea behind transfer learning is to train on a related task to the end task first. Then the network weights in the model for the actual task are initialized to those of the model we are transferring from, so the training of the original model is a pre-step to the real task. Since training the models on ImageNet scale datasets is not generally feasible due to their large number of parameters and long training time, one of the pre-trained models is picked and then fine-tuned. Fine-tuning a classifier means taking the examples for the final task, and training the network on those, updating the original classifier at the same time. The transfer learning approach differs significantly from the traditional learning model, where each task requires a separate model that learns from the given data using random weights. Since image inputs are often very high dimensional, the traditional approach may not work in many cases. The pre-training allows for focusing on data that provides answers to the actual task and not on learning low-level features, which the ImageNet classifiers would already have learned.

\begin{figure}[h!] 
\centering 
\includegraphics[width=0.8\textwidth]{imgs/imagenet_parameters.png}
\caption{Number of parameters in popular ImageNet classifiers. Figure from \citep{efficientNet}.\label{fig:params}}
\end{figure}

Picking which classifier to use as a base is often problem-dependent. As it is not possible to declare one network structure to be the best at all tasks, picking the best model to start with usually requires the user to compare different architectures and weighing the requirements for the problem at hand. Often though, the larger models will perform better, and there exists a correlation between performing well on ImageNet and being a good transfer learning model \citep{betterTransfer}. Though as can be seen in figure 3.1, better performance often comes at the cost of many more parameters, requiring more memory to train the model. Though just the number of parameters is not the only thing to compare as the throughput of a Resnet50 turns out to be about three times as large as the throughput of an EfficientNet-B4 even though they have a similar amount of parameters \citep{classifierPerformance}. So in the case of time-constrained inference environments, also the time required for an inference has to be taken into account when picking models.

If there is enough data, it turns out that using a pre-trained network does not provide any benefits in terms of the converged model accuracy, but it is not detrimental to performance either \citep{rethinkTransfer}. When training sufficiently long on a sufficient amount of data, the pre-trained and randomly initialized networks converged to similar accuracies but required significantly different amounts of training resources. Still, this does not mean that pre-training is useless by any means as the saved resources and getting models to converge faster are essential factors for progress, and of course, in many cases, training from scratch will not provide satisfying results.

\section{ResNet}
ResNet is one of the first effective and very deep Convolutional Neural Network architecture that was presented in 2015 and won the ImageNet challenge. Prior to the publication of ResNet, the most powerful networks were relatively shallow, like VGGNet, which has only 19 layers. One of the big issues relating to training deep networks is vanishing gradients, where gradients disappear when they are backpropagated through many layers \citep{wideResNet}. ResNets utilize residual connections around bottleneck building blocks, which allow for the networks to contain many more layers than those without them, the largest network presented in the original ResNet paper was 152 layers, totaling for around 8x increase in the number of layers when compared to earlier networks \citep{resNet}.

\begin{figure}[h!] 
\centering 
\includegraphics[width=0.8\textwidth]{imgs/resnet-block.png}
\caption{Resnet building block \citep{resNet}}
\end{figure}

The ResNet block architecture allows for the blocks to learn the identity function more efficiently by trying to learn the residual function instead of the direct mapping. Normally a mapping is learned between ${x}$ and ${y}$ using a function ${H(x)}$ but by the same token, we can learn a residual transformation, ${F(x) = H(x) - x}$, where ${H}$ is the mapping of two or more network layers. Both of these approaches should approximate the same functions; the difference is in how easy it is for the network to learn. Learning a transformation of ${F(x) = 0}$ would intuitively seem easier for a neural network than learning ${F(x) = x}$. As can be seen from the success of the ResNets compared to non-residual networks, this is what allows for creating very deep networks. If the transformation changes the input size, a matrix ${W}$ is necessary to map the input to the same dimensions, generating the final formula for a residual block ${y = F(x, \{W_i\}) + W_s x}$.

Many variants of ResNet exists, such as wide ResNets \citep{wideResNet}, ResNeXt \citep{resNext} and others. Also the DenseNet \citep{denseNet} is heavily inspired by ResNets. ResNet-50, ResNet-34, ResNet-101, and ResNet-152 are still some of the most popular models to use when a pre-trained ImageNet trained backbone is needed for some part of a classifier as they produce good results and do not contain too many parameters compared to some other architectures.

\section{Improving model performance}
Improving model performance is not easy, but using the various types of skip-connections, as the ResNet residual blocks use, it is possible to increase the size of the network to massive sizes. A 557 million parameter model called GPipe \citep{gPipe} takes the model scaling to the extreme and requires some unique parallelism libraries to train the model. It is still only slightly better than previous models, showing that size is not the only thing that matters.

\begin{figure}[h!] 
\centering 
\includegraphics[width=0.8\textwidth]{imgs/scaling-networks.png}
\caption{Various ways of scaling network architectures \citep{efficientNet}}
\end{figure}

There are three main ways to scale up a network, as shown in figure 3.3. Scaling by depth means adding more layers to the model, allowing for more complex dependencies to be captured by the model. For example, ResNet-1000 is a very deep type of ResNet, but it has similar performance to a ResNet-101, so there are diminishing returns when trying to scale up by depth \citep{efficientNet}. Scaling by width means increasing the number of channels in the layers, and it is especially popular when optimizing smaller size models, such as MobileNet \citep{mobileNet}. Wide ResNets \citep{wideResNet} increase the width of the ResNet blocks and allow for better features and easier training.

\section{EfficientNet}
EfficientNet models form a family of models ranging from EfficientNet-0 to EfficientNet-7 that were generated by smartly scaling existing convolutional models to optimize them for efficiency \citep{efficientNet}. The more complex models are created by using compound scaling on the EfficientNet-0 model, where the width, depth, and the resolution of the network are scaled using a factor ${\phi}$. The search of ${\phi}$ is an optimization problem, where the goal is to optimize for both accuracy and number of floating-point operations. This search is only done on the base model because the neural architecture search used to find the parameter becomes very expensive as the model size increases \citep{efficientNet}.

The proposed EfficientNet models use mobile inverted bottleneck (MBConv) blocks used in MobileNetV2 \citep{mobileNetv2} in constructing the base model. The MobileNet uses a Depthwise Separable Convolution, where a depthwise convolution and a pointwise convolution are applied in sequence. The depthwise convolution applies ${d_i}$ ${k x k}$ filters to the input, where ${d_i}$ is the input channels and ${k}$ is the kernel size leading to an output channel count ${d_j = d_i}$. A normal convolutional layer would apply multiple filters having ${M}$ channels with the computational cost of ${h_i w_i d_i d_j k^2}$, whereas the depthwise convolution only has a cost of ${h_i w_i d_i (k^2 + d_j)}$, reducing the cost by ${k^2}$. The result of the depthwise convolution then runs through a pointwise convolution, where a 1 x 1 x ${d_j}$ 1d convolution is applied to get the final output as a linear combination of the channels.
\chapter{Multi-task learning}
Multi-task learning is a generalization of Transfer Learning to training a single model where the goal is to optimize for multiple objectives.
Training a Multi-task model requires a set of distinct tasks to be posed as a Multi-task learning problem.
It can have many benefits when applied correctly, such as making the models more general by regularization and reducing computational requirements.
On the other hand, if Multi-task learning is applied to problems that don't train well in a Multi-task setting, the resulting models can be significantly worse than the Single-task counterparts.

\section{Definition}
Multi-task learning is quite similar to Transfer learning.
However, the main difference lies in the fact that the goal is to generalize to solve and improve performance in all tasks using some shared representation.
In contrast, Transfer learning aims to optimize a single new task, ignoring the performance on the original entirely.
Often the shared layers are initialized using some ImageNet model and Transfer learning as the ImageNet backbone is an architectural feature shared in many models.

The formal definition for Multi-task learning follows the definition given in \citep{surveyOnMultiTask}.
A learning problem consists of ${m}$ related learning tasks ${\{T_i\}_{i=1}^m}$ that are trained together.
Each task has a dataset ${D_i}$ with ${n_i}$ pairs ${\{x_{j}^{i}, y_{j}^{i}\}_{j=1}^{n_i}}$, where ${x_{j}^{i}}$ refers to the input of task ${T_i}$ and ${y_{j}^{i}}$ is the label corresponding to the input vector.
Let ${X^i}$ = (${x_{1}^{i}, ... , x_{n_i}^{i})}$ be the input matrix for task ${T_i}$.
In a Multi-task setting there can be ${x}$ such that ${x \in X^i}$ and ${x \in X^j}$ or ${X^i = X^j}$ for some ${i \ne j}$, meaning that a single image has multiple labels or an entire data set has labels for multiple tasks.

To train the network, each task ${T_i}$ needs to have its loss function defined.
A commonly used way to get the total loss of the input by using the weighted loss functions for all tasks, resulting in a formula ${L_{total} = \sum_i{w_i L_i}}$, where ${L_i}$ is the loss function for task ${T_i}$ and ${w_i}$ is the weight that specifies the sensitivity of the task \citep{usingUncertaintyToWeighLosses}.
The sensitivity of the tasks is an essential parameter to get right as choosing a too high parameter for some task might lead to a solution that is optimal for only the most sensitive task, starving the others \citep{whichTasks}.

Once the loss function is specified, the actual training is quite similar to training a Single-task model.
The one new thing to consider in a Multi-task setting is the sampling ratio of the different tasks as some tasks can be easier to learn or contain significantly more or less data compared to other tasks.
Like in regular training, the models are trained one batch at a time.
In a Multi-task setting, we have datasets for each of the tasks from which we can pick at a time.
Now, we can define an epoch in the training process as the total number of batches from all tasks.
We can define this as $\sum_i{ \dfrac{|X_i| \alpha_i}{B_i}}$ batches over $i$ tasks, where $|X_i|$ is the training set size, $B_i$ is the batch size, and $\alpha_i$ is the scaling factor for each task, telling how important the specific task is.
Iterating through all batches can be done in a round-robin fashion, by going through all batches of each data set at a time or randomly sampling the sets with probabilities respective to their magnitude and scaling factor.

\section{Latent image representations}
For a Multi-task model, the optimal situation is such, where all classifiers would use a single shared latent representation of the image to predict all attributes.
This representation is an Image Embedding, that would capture all valuable information of the image for our tasks.
The embedding could be the vector represented by the final layer of an ImageNet classifier flattened to some n-dimensional vector that then solves all the tasks instead of just a single task.
\begin{figure}[h!]
    \centering
    \includegraphics[width=0.8\textwidth]{imgs/sharedBackbone.png}
    \caption{Example of partly and completely shared backbones. Figure from \citep{visualPerson}.\label{fig:params}}
\end{figure}
However, such a universal image representation can be challenging or impossible to learn, and instead, only a part of the network is shared between the tasks.
Here, this shared structure is called the backbone of the Multi-task classifier.
Often the backbone is in the form of some ImageNet classifier architecture, and it can be the entire classifier or some layers up to some arbitrary limit.
As the amount of layers shared is picked for each task separately, some tasks can share the entire classifier as a backbone, and others might only share some amount of layer that produces desirable results.

Besides just deciding the network branching, each task requires an independent head that outputs prediction for that task.
These task heads are similar to those used in transfer learning, where only a single task is solved using the ImageNet backbone.
Especially in a Multi-task setting, these heads can be more than just a single linear layer calculating the softmax over all classes.
A single head can contain any combination of several linear layers, dropout layers, batch normalization layers, convolutional layers, or it could be a Recurrent Neural Network creating a caption for the image embedding, depending on the task at hand.
In Figure 3.1, we can see an example of the flexibility of the sharing as the model can use a completely shared backbone or a branching backbone, which has a unique head for each task, solving the person re-identification and classification tasks.

\section{Benefits}
Depending on how and where Multi-task learning is applied, it can provide a multitude of benefits to the model.
Many of these benefits stem from the fact that the model gets to see more data, and the various tasks can make it easier to find useful features.
Different kinds of data sets have different kinds of noise, learning multiple tasks makes it easier to distinguish which features are beneficial and detrimental some features may be difficult to learn on a dataset but can be useful and borrowed from another and learning multiple tasks forces the model to not overfit on one of them \citep{ruderOverview}.

These benefits come up when the tasks are compatible and allow the model to learn more general features, generally leading to better performance on the distinct tasks.
When the needed features between tasks are conflicting, the model performance tends to go down \citep{uberNet}.
The problem of deciding what to share is not easy to solve.

The benefits of Multi-task learning come up, especially when dealing with limited amounts of data, in which case, particularly finding the features that matter without overfitting can be complicated.
For example \citep{biologicalMultitask} found that Multi-task learning could improve gene expression pattern classifiers when trained in a Multi-task setting.
The Multi-task classifier was significantly better than the one only using Transfer learning, showing that the features learned were more general.

Another significant boon of Multi-task models, especially in embedded domains, is the reduction in model size and inference time.
As many classifiers are dependent on an embedding of an image to produce results, using a shared embedding of an image for multiple tasks means that the model requires only a single partly or completely shared backbone.
For example \citep{multiPoseNet} uses a shared backbone to detect people in an image and to detect keypoints on their body and then finally to do a semantic segmentation of the image while also improving performance on most of the tasks.

Finally, a model using Multi-task learning can take advantage of the different losses to produce a more optimal loss weighing strategy compared to constant weighting.
Picking the weights in the total loss function is very important as with invalid weights, the optimization can be difficult or even impossible \citep{lossWeighting}.
The weights become even more critical if the losses for various tasks are different, for example, one task might use mean squared error as a loss function and another cross-entropy, and the resulting loss values might differ by orders of magnitude.
The total loss function can be modified by adding an uncertainty weighting to each of the tasks by considering the uncertainty of the prediction \citep{usingUncertaintyToWeighLosses}.
Depending on the task, the benefit compared to an unweighted loss can be significant \citep{usingUncertaintyToWeighLosses} or, in some cases, only small \citep{lossWeighting}.

\section{Hard parameter sharing}
A hard parameter sharing setting is one where the same convolutional weights in the intermediate layers solve multiple tasks, like in Figure 3.1.
Hard parameter sharing is the primary way of reducing total model size and inference time when solving multiple tasks simultaneously.
While sharing the weights can provide many benefits to the end model, as seen in the previous chapter, it comes at the cost of an increased number of tunable parameters when training the models.
As we have previously seen, Multi-task models introduce new constant factors to the training process beyond the normal hyperparameters for learning rate, dropout rate, and others.
These include the hyperparameters for weighing the loss functions and determining the task sampling ratios, but also the expensive to evaluate architectural decision of which tasks should share representations and how much should be shared.

\begin{figure}[h!]
    \centering
    \includegraphics[width=1\textwidth]{imgs/multipleShares.png}
    \caption{Many ways of sharing between tasks. Figure from \citep{healthyDrink}.\label{fig:params}}
\end{figure}

In Figure 3.2, there are multiple ways of configuring a Multi-task network in terms of what to share.
The results for the different configurations were quite varied.
However, even the worst Multi-task Model A did better on every task when compared to the single-task models, and accuracy improvement from the single-task case to the best multi-task Model F was roughly 20\% \citep{healthyDrink}.
These results are quite validating for the presumed benefits of Multi-task models.
Since these results show that it is possible to share the majority of the network parameters and, at the same time, increase the performance, even with the most greedy share-all model.
Though, at the same time, it shows that multiple compute-expensive experiments have to done to find the most optimal sharing structure for the backbone network.

The two original features of interest were sugar and alcohol level in the drinks, and the other features were added as auxiliary training features to improve the performance \citep{healthyDrink}.
This auxiliary training can also be done by, for example, using ImageNet to keep the original classifier accurate on ImageNet while training on the new data set like in \citep{biologicalMultitask}; this way, the model won't be able to overfit on the new data so easily.
In the drink classification experiment, the auxiliary tasks were picked in a way that they might help in finding the correct features to solve the actual tasks \citep{healthyDrink}.
Adding auxiliary tasks is an interesting way of forcing features into the model that the developer thinks could be useful for the model to find, in that sense, they are hand-crafted features that get augmented in the model.
In practice, this means that if we are training a model to predict the steering angle, we might want to add an auxiliary task that predicts the lane markers to make it easier to learn these features.

A similar Multi-task model in \citep{visualPerson}, also partly visualized in Figure 3.1, solves 6 different tasks in a single model.
The tasks in that model are very different from one another, including person re-identification, pose estimation and image segmentation.
The interesting result in their search for related tasks is that while a pair of tasks don't work well together, they can work well when combined with other tasks.
For example, they found that just pairing the pose estimation head with other tasks significantly reduces its accuracy, but when all tasks use the single shared backbone, it does just as well as if trained in a single-task setting.
These results show that very different tasks can combine to a single model to provide a very significant reduction in model size while producing about as good or better accuracies.

\section{Soft parameter sharing}

\begin{figure}[h!]
    \centering
    \includegraphics[width=0.8\textwidth]{imgs/stitch.png}
    \caption{Cross-stitch architecture. Figure from \citep{crossStitch}.\label{fig:params}}
\end{figure}

Soft parameter sharing is quite similar to learning each task on its own since each task has separate weights.
Soft parameter sharing happens by picking some layers, where some metrics constrain the parameters to be similar, like ${l_2}$ distance \citep{ruderOverview}.

The sharing functionality can be more complicated than just basic regularization.
For example, the Cross-stitch units \citep{crossStitch} are a particular type of soft sharing, where the stitch units combine tasks A and B using a linear combination of the activations, visualized in Figure 3.2.
The benefit here is that the user does not need to specify how much should be shared between the tasks, but it is instead learned in the cross-stitch unit.
If nothing needs to be shared, then the network can learn to assign the weight for the other task to zero.

This kind of sharing does not scale very well as the number of tasks increases since calculating the relation between various tasks is quadratic.
Since the joining logic requires extra parameters, it means there is even more to learn than just learning all the tasks separately.
As soft sharing does not provide many of the benefits that are gained by hard sharing the parameters, it is not as popular, but still used in various cases.
However, it is useful in adding some extra information between the tasks to gain extra performance by utilizing the latent representations between them in various ways.
Also, in the basic case of using a soft parameter share to constraint, the layers can improve the generalizability of the model.

\section{Attention augmented Multi-task learning}
In a Multi-task setting, the model ends up having a representation of the image from each task's point of view, and it is possible to use attention \citep{attention} to augment the predictions.
Especially in the case where the tasks are heavily correlated, this can be quite useful.
For example, when doing weather recognition, it would make sense that some of the features are correlated, and this correlation can be used by adding an attention layer between the different tasks, like in \citep{cnn-rnn}.
The Multi-Task Attention Network (MTAN) \citep{multiTaskAttention} is an example of a more advanced application of attention in Multi-task models.
In the MTAN, each task gets its own attention module, and they are used to determine correlations of the tasks at specific layers of the network using the attention operations.
It is similar to cross-stitching networks in that each task has to learn its attention module features.
However, there is only a single backbone for the shared features, and the attention modules are responsible for producing the task level outputs.
The number of parameters is still significantly reduced as the task specific-parts are not complete ImageNet classifiers, resulting in a significantly more efficient and accurate classifier for image segmentation.

\section{What and when to share}
What makes or breaks a Multi-task model is the decision on what should be shared, but determining what and how much should be shared is exponentially more expensive as the number of tasks grows.
As experimentation is expensive, a good starting point is to try to find similar tasks that should do well together.
However, there is no guarantee that even all the tasks within a single family of problems are beneficial for joint training, so some experimentation has to be done.
The desired result in a Multi-task setting would be to share all the parameters between all the tasks, but as can be seen in \citep{uberNet}, that approach often significantly reduces performance.
If sharing an entire network does not produce good results, it can be a good idea only to share some part of it as the lower level features tend to be more general \citep{transferringMidLevelRepresentations}.
By sharing only a part of the network, it may be possible to strike the right balance between performance and model complexity.

\begin{figure}[h!]
    \centering
    \includegraphics[width=0.8\textwidth]{imgs/taskonomy.png}
    \caption{The taskonomy task similarity tree. Figure from \citep{taskonomy}.\label{fig:params}}
\end{figure}

An attempt to create a taxonomy of related tasks called Taskonomy to find out what tasks are beneficial for transfer learning exists in \citep{taskonomy}.
The Taskonomy experiments pre-trained networks on tasks and then evaluated whether the features would transfer well to other tasks to end up with a hierarchical categorization of related tasks, shown in Figure 3.2.
Still, the authors noted that, depending on model architecture and data set, the results could be different.

It would make sense that the results of the Taskonomy would be easily transferable to the Multi-task setting.
When evaluating the Multi-task vs. Transfer affinity, it turns out that they are negatively correlated, at least in the case of the five tasks that were the focus of the experiments in \citep{whichTasks}.
Based on this observation, it can be beneficial to train some non-related tasks together.
The authors suggest that the different tasks act as a good form of regularization as the models need to generalize to multiple types of inputs.
It could also be that some of the learned features work very well for the other task, but can't be easily learned with the dataset of the other task and vice versa.
These results are aligned with the empirical results that we saw when we looked at the compatibility of classifiers using hard weight sharing.
Currently, it seems not to be possible to determine the compatibility of different tasks purely on a theoretical grounding.
Instead, experiments need to be run, and the results of pairs of tasks do not always generalize to a set of all other tasks when using shared representations in models \citep{visualPerson}.

\chapter{Object detection}
Object detection is another prevalent task in the domain of computer vision.
An object detection task is one where the goal is to localize one or many different classes of objects using bounding boxes.
Before current deep learning-based techniques, a popular way to solve the problem was to use handcrafted features like in image classification and to use a sliding window over the image for localizing objects.
These previous techniques include, for example, the Viola-Jones detector \citep{viola-jones}, which uses a sliding window and AdaBoost for features, and another popular method was using histograms of oriented gradients \citep{hogs} to find where the boundaries of objects exist.
These days there exist various ways of using the feature maps of neural networks for learning the filters to get much better results.
The various architectures can be split into two approaches, one-stage detectors, and two-stage detectors.
The two-stage detectors require region proposals, based on which the object detection is done.
An often-used family of these kinds of methods is the R-CNN classifiers, for example, the faster R-CNN \citep{faster-rcnn}.
In the one-stage detectors, features from various layers of the classifier are used to get the predictions.
For example, the YOLOv4 \citep{yolov4} is the 4th iteration of the single-stage YOLO family of detectors that are very popular due to the good balance between fast speed and good accuracy.

\section{Metrics}
The training of object detection models requires data sets that have been labeled for that purpose.
Generally, the data sets contain bounding box annotations for each of the classes in the data set, such as seen in Figure 5.1.
For this reason, object detection models can't use a simple metric like the basic accuracy in image classification.
As the images are often manually labeled, the boxes are most likely not completely consistent.
So most likely, the predictions are never going to align with the labeled boxes perfectly.
Consequently, the method for evaluating object detection performance is Average Precision (AP) or mean Average Precision (mAP) or one of their variants.
These metrics use intersection over union (IoU) score to evaluate how incorrect the predicted bounding box is when compared to the actual label. Intersection over union is visualized in Figure 5.1.

\begin{figure}[h!]
    \centering
    \includegraphics[width=0.8\textwidth]{imgs/intersection_over_union.png}
    \caption{The visual formula for calculating intersection over union.}
\end{figure}

To get to a final AP score, the first thing that has to be decided is the IoU score threshold for considering a prediction to be correct.
For example, we could consider all predictions that have over 50\% or 75\% overlap in the IoU.
The threshold value for IoU that should be used is not entirely standardized, and there are multiple ways of calculating the AP score.
For example, the popular COCO dataset \citep{COCO} uses an average of 10 IoU scores ranging from .5 to .95 IoU thresholds as the main metric.
To get the AP for a class, we need to graph the precision-recall curve and then calculate the area under it.
For example, the Pascal Visual Object Classes Challenge (VOC) \citep{PVOC} recommends doing this by using 11 points of interpolated precision.
So we get the following formula for Average Precision:

\[AP = \frac{1} {11} L_{total} = \sum_{r \in \{0, 0.1, \ldots, 0.9 , 1\}}{p_{interp}(r)}\] \noindent

Where $p_{interp}$ is defined as

\[p_{interp}(r) = \max \limits_{\tilde{r}:\tilde{r} \geq r} p(\tilde{r})\]

Then to get the mAP score, we can average the AP score for each of the classes in the data set.
As was mentioned, this way of calculating the AP is not always the same, for example, the COCO metrics \citep{COCO_SITE} recommend using 101 points for integrating the curve instead of the 11 proposed in the VOC.
This discrepancy in the integration means that not all AP scores are directly comparable.

\section{Multi-dataset training}
Often an object detection problem requires detecting multiple different classes in a similar context.
For example, a self-driving car would need to detect various traffic signs, cars, pedestrians, road markings, cyclists, and many other things.
Collecting a single dataset that has labels for all of the classes of interest can be a very daunting task.
Even if it is feasible to create the dataset for the original classes of interest, this approach does not scale very well when new classes need to be recognized.
As likely the original dataset might contain millions of images labeled for multiple classes, adding a new class would require going through the entire dataset again and labeling the new class as well.
The new class may be relatively rare; for example, we may be interested in detecting emergency vehicles with sirens on.
Here is where multi-dataset learning is highly beneficial as it only allows for collecting a specific class dataset.
This type of cross-dataset learning is useful when we need to combine multiple distinct data sources to detect some union of the labels \citep{cross_data}.

The main difficulty in combining multiple datasets for detection lies in the fact that they likely contain unlabeled overlapping classes.
For example, given a dataset for detecting cars and another for detecting stop signs, we have two distinct datasets where only one of the classes is labeled.
If we train this model, assuming that the labels are genuinely valid, we will end up unlearning the tasks due to the overlap of the classes.
The problem lies in the fact that it is most likely that in the car dataset, we will find stop signs that are not labeled.
Similarily we will find cars that are unlabeled within the stop sign dataset.
When we naively train this model, we will end up detecting cars and stop signs that are actually correct but incorrect based on the labels.
The model will be punished for detecting these non-labeled positive examples due to the absence of the labels.

Similar multi-dataset training can also be applied in multi-label image classification settings.
A multi-label image classification problem is one where we want to assign multiple labels to an image.
For example, we might want to classify whether it is raining, the sun is shining, the sea is visible, is there a dog in the image, and so on for all the classes of interest.
Again, collecting a dataset containing all labels for all images is quite expensive, but we can train this with a separate binary classification head for each task.
And as collecting a separate dataset for each task is relatively easy, it is possible to create a quite powerful model with relative ease.

\section{Self-supervised object detection}

\section{EfficientDet}

\chapter{Datasets}
Describing our data
\section{Data formats}
Describe what datasets are used for each of the tasks and where they are from etc.

\chapter{Experiments}
Here are our experiments
\chapter{Conclusions}
In this thesis, we have given an overview of fundamental computer vision tasks of image classification and object detection.
We described the importance of the ImageNet data set and some of the most significant image classification architectures based on it.
The importance of these models extend beyond just solving the ImageNet dataset, and we covered how it is possible to apply the trained models in new classification problems with transfer learning.

For object detection, we described how the problem and datasets differ from image classification ones.
We also described the different parts of object detectors and how single and two-stage detectors differ from one another and when one might be preferred over the other.
The object detectors also show how important the ImageNet backbones are, as they are an integral part of the object detection architectures also.
The focal loss is an important loss function used by modern single-stage object detectors in allowing a large number of anchor points for localizing the objects in the images.
Finally, we described one of the current state-of-the-art object detection architectures, the EfficientDet, and how it can be trained on a new problem using transfer learning like in the case of image classification.

Here we have also presented multi-task learning as a generalization of transfer learning.
In a multi-task setting, the goal is to solve multiple tasks using some shared representation. 
It differs from transfer learning in that all tasks are optimized continuously, rather than just serving as a starting point.
Doing transfer learning is an excellent way of obtaining good results, even on little data.
With multi-task training, we can utilize the common model architectures as shared parameters for multiple tasks to allow the model to generalize better and run more efficiently.
This is possible because the model is required to use the same capacity to solve multiple problems that may require different features, some of which can also be useful in other tasks. 
Still, sometimes the multi-task setting can be detrimental to accuracy.
Despite the differences in performance, utilizing shared representations means that there are fewer parameters that the model needs to learn, leading to smaller models and shorter inference times.

We saw multiple examples where multi-task learning played an important part in getting models that are efficient and performant.
One of these was the object detection, where the model tries to classify and localize the objects.
Both soft and hard parameter sharing for multi-task learning were covered, with some specialized architectures taking advantage of the fact that multiple tasks are solved at the same time.
Finally, we found that multiple tasks can serve as auxiliary targets for training to aid the model in finding important features.

A multi-task problem setting modifies the training process and introduces some new parameters that can be tuned when training models.
Task-specific sampling ratio during the data sampling process and giving task-specific loss weights were the two new factors that need to be tuned when different tasks are combined.
These weights need to be tuned appropriately for the model to be able to find the features that work for all tasks.
Beyond just tuning parameters, multi-task models may need some architectural fine-tuning in deciding which tasks should have shared representations and just how much should be shared.

The training process differences due to the multi-task problem setting were evaluated through multiple experiments on combining various datasets into a single model.
With these experiments, we empirically verified some of the assumptions of training the models in a multi-task setting.
We noted that it was difficult to find the very best models as different sampling ratios resulted in very different final models.
As the model gets larger and the number of tasks increases, finding the correct ratios gets ever more difficult.
After adding new tasks, the previous ratios might also need adjusting.
Also, we discussed the difficulties of modifying the existing models to contain our desired output heads.
This was most apparent in the case of object detectors. 
The supplied code is relatively pre-packaged to run on the specified training loop, and modifying these model definitions to add new tasks takes some effort.
We saw some improvements and reductions in the model performance, but in general, the multi-task models were close to the single-task counterparts so that reducing the model size might be a good trade-off.

In the end, what we learned is that the success of multi-task learning some tasks can't be pre-determined currently, and some expensive experimentation is needed to arrive at the best multi-task architecture.  
Thus, multi-task learning can be considered as a trade-off of training time complexity for inference-time benefits in accuracy and speed.
% \chapter{Introduction}
\section{Problem statement and motivation}
The following gives some superficial instructions for using this template for a Master's thesis. For guidelines on thesis writing you can consult various sources, for example, the Bachelor thesis template.

The thesis should have an introduction chapter. Other chapters can be named according to the topic. In the end, some summary chapter is needed; see Chapter~\ref{chapter:conclusions} for an example.

\section{Structure of thesis}
Cool stuff etc.

\chapter{Neural networks}
Background stuff 
\section{Fully Connected Networks}
When needed example of a transfer learning head?, when good/bad
\section{Convolutional Neural Networks}
Show some basic CNN things maybe and why needed over fully connected bois :) 
\subsection{Neural image processing}
How it different from classical features and how/why it works
\section{Attention}
Attention huh
\section{Multi-task learning}
Describe how a multi-task setting differs or maybe in the other chapter only?
\section{RNN/LSTM/Transformer text processing/generation}
Depending on what may be used










\chapter{Image classification}
Image classification is one of the essential modern computer vision problems, where the goal is to create a model that can classify an input image into one of a set of pre-defined classes. Before the popularization of applying large Convolutional Neural Networks for this task, the most successful way of solving the problem was to use some algorithm for finding feature descriptors in a set of images to construct a Visual Bag of Words.  A linear classifier, like a Support Vector Machine, would then do the classification using the Bag of Words representation of an image. These days nearly all approaches are based on using deep CNNs, and working CNNs were deployed already in 1998 on character recognition in the form of LeNet \citep{leNet}.

\section{ImageNet}

ImageNet  \citep{imagenet}  is perhaps the most significant dataset for image classification and especially the ImageNet Challenge \citep{ILSVRC}, which is a challenge for a collection of 1000 classes from the ImageNet dataset for image classification using 1.2 million training and 150 thousand test images of the entire ImageNet dataset. In 2012 the winning model, AlexNet \citep{alexNet}, showed that it was possible to train a deep CNNs efficiently using GPUs. Since 2012 all top-performing models showed some new improvements on how to create the most performant network architecture, for example, VGGNet from the year 2014 and ResNet \citep{resNet} from 2015, both of which have been popular models to use for Transfer Learning since.

Human accuracy on the ImageNet challenge is about 5.1\% \citep{imageNet_summary}, and ResNet achieves a top-5 error rate of 3.57\% \citep{resNet} and newer architectures even lower, but this still does not mean that image classification is a solved problem. The human performance experiment found that many of the human errors are caused by not having expert information in, for example, identifying animal species or not even being aware of the existence of a class \citep{imageNet_summary}. ObjectNet \citep{objectNet} is a dataset designed to test image classifiers with a focus on their generalizability. It contains many classes that also exist in the ImageNet dataset. However, they are in unexpected locations or have an unexpected pose, causing the high accuracy image classifiers trained on ImageNet to experience a 40-45\% accuracy drop when evaluating them on the ObjectNet images of classes shared between ImageNet and ObjectNet. This kind of adjustment is relatively easy for a human, and it shows that while the classifiers are good, they are by no means perfect.

\section{Transfer learning}
Transfer learning is a powerful technique to obtain results quickly when using deep CNNs. Here we will only take a look at transfer learning within the domain of deep CNNs, but it is a technique that has been successfully applied to many other domains of machine learning as well.

To give a formal definition of transfer learning, we will follow the definitions provided in \citep{transferSurvey2010}. Let $\mathcal{D}$ be a domain, which consists of a feature space $\mathcal{X}$ and a marginal probability distribution P($\mathcal{X}$). For a given domain, a task $\mathcal{T}$ = {$\mathcal{Y}$, f$\mathcal(X)$} consist of a label space $\mathcal{Y}$ for the inputs and of a predictive function $f(x)$ which produces predictions for all pairs ${x_i, y_i}$ where $x_i \in \mathcal{X}$ and $y_i \in \mathcal{Y}$. Given a source domain $\mathcal{D}_S$, a source task $\mathcal{T}_S$, a target domain $\mathcal{D}_T$ and a target task $\mathcal{T}_T$, where target and source are disjoint, transfer learning tries to improve the performance of $f_T(x)$ using $\mathcal{D}_S$ and $\mathcal{T}_S$.

Unlike the ImageNet challenge, most real-world tasks do not have such an abundance of data for all possible classes. Still, to achieve the highest accuracies, they require models that are equal in terms of complexity to those that have top accuracies problems on the scale of the ImageNet classification. For this reason, many CNN classifiers, irrespective of the problem, feature one of the ImageNet classifiers in the model architecture. Even though some datasets may contain a large number of images per class, using a pre-trained classifier as a basis often produces a better final classifier by applying fine-tuning \citep{betterTransfer}.

The idea behind transfer learning is to train on a related task to the end task first. Then the network weights in the model for the actual task are initialized to those of the model we are transferring from, so the training of the original model is a pre-step to the real task. Since training the models on ImageNet scale datasets is not generally feasible due to their large number of parameters and long training time, one of the pre-trained models is picked and then fine-tuned. Fine-tuning a classifier means taking the examples for the final task, and training the network on those, updating the original classifier at the same time. The transfer learning approach differs significantly from the traditional learning model, where each task requires a separate model that learns from the given data using random weights. Since image inputs are often very high dimensional, the traditional approach may not work in many cases. The pre-training allows for focusing on data that provides answers to the actual task and not on learning low-level features, which the ImageNet classifiers would already have learned.

\begin{figure}[h!] 
\centering 
\includegraphics[width=0.8\textwidth]{imgs/imagenet_parameters.png}
\caption{Number of parameters in popular ImageNet classifiers. Figure from \citep{efficientNet}.\label{fig:params}}
\end{figure}

Picking which classifier to use as a base is often problem-dependent. As it is not possible to declare one network structure to be the best at all tasks, picking the best model to start with usually requires the user to compare different architectures and weighing the requirements for the problem at hand. Often though, the larger models will perform better, and there exists a correlation between performing well on ImageNet and being a good transfer learning model \citep{betterTransfer}. Though as can be seen in figure 3.1, better performance often comes at the cost of many more parameters, requiring more memory to train the model. Though just the number of parameters is not the only thing to compare as the throughput of a Resnet50 turns out to be about three times as large as the throughput of an EfficientNet-B4 even though they have a similar amount of parameters \citep{classifierPerformance}. So in the case of time-constrained inference environments, also the time required for an inference has to be taken into account when picking models.

If there is enough data, it turns out that using a pre-trained network does not provide any benefits in terms of the converged model accuracy, but it is not detrimental to performance either \citep{rethinkTransfer}. When training sufficiently long on a sufficient amount of data, the pre-trained and randomly initialized networks converged to similar accuracies but required significantly different amounts of training resources. Still, this does not mean that pre-training is useless by any means as the saved resources and getting models to converge faster are essential factors for progress, and of course, in many cases, training from scratch will not provide satisfying results.

\section{ResNet}
Residual man 
\section{EfficientNet}





\chapter{Object detection}
Based on what is going to be detected, maybe text, cars, people?
\chapter{Paragraph captioning}
Assuming we are doing this
\chapter{Multi task learning}
Cover some examples of MTL in Computer Vision, differences to Transfer Learning, why it might work, why we might see detrimental results and why it might be beneficial (results/performance)

\chapter{Datasets}
Describing our data
\section{Data formats}
Describe what datasets are used for each of the tasks and where they are from etc.

\chapter{Experiments}
Here are our experiments
\section{Training process}
\subsection{Evaluation criterias}
eval
\section{Image classifiers}
Results of image classsifiers on the datasets maybe compare ResNet vs EfficientNet
\subsection{Only transfer learned classifier}
Just a ResNet or something
\subsection{Image classifier with attention}
Add attention to image classifier
\subsection{Classifiers combined to a multi-task model}
How it do
\section{Object detection}
Some stuff for object detection
\section{Multi-task models}
Create multi-task models for all and try to find some pairs/tuples that actually work together properly
\chapter{Result analysis}
Incredible models absolutely.

\chapter{Future work}
How to make it better what could be tried 

\chapter{Figures and Tables}

\section{Figures}
Figure~\ref{fig:logo} gives an example how to add figures to the document. Remember always to cite the figure in the main text.

\begin{figure}[h!] 
\centering 
\includegraphics[width=0.3\textwidth]{HY-logo-ml.png}
\caption{University of Helsinki flame-logo for Faculty of Science.\label{fig:logo}}
\end{figure}

\section{Tables}

Table~\ref{table:results} gives an example how to report experimental results. Remember always to cite the table in the main text. 

\begin{table}
\centering
\caption{Experimental results.\label{table:results}}
\begin{tabular}{l||l c r} 
Experiment & 1 & 2 & 3 \\ 
\hline \hline 
$A$ & 2.5 & 4.7 & -11 \\
$B$ & 8.0 & -3.7 & 12.6 \\
$A+B$ & 10.5 & 1.0 & 1.6 \\
\hline
%
\end{tabular}
\end{table}

\chapter{Citations}

\section{Citations to literature}

References are listed in a separate .bib-file. In this case it is named \texttt{bibliography.bib} including the following content:
\begin{verbatim}
@article{einstein,
    author =       "Albert Einstein",
    title =        "{Zur Elektrodynamik bewegter K{\"o}rper}. ({German})
        [{On} the electrodynamics of moving bodies]",
    journal =      "Annalen der Physik",
    volume =       "322",
    number =       "10",
    pages =        "891--921",
    year =         "1905",
    DOI =          "http://dx.doi.org/10.1002/andp.19053221004"
}
 
@book{latexcompanion,
    author    = "Michel Goossens and Frank Mittelbach and Alexander Samarin",
    title     = "The \LaTeX\ Companion",
    year      = "1993",
    publisher = "Addison-Wesley",
    address   = "Reading, Massachusetts"
}
 
@misc{knuthwebsite,
    author    = "Donald Knuth",
    title     = "Knuth: Computers and Typesetting",
    url       = "http://www-cs-faculty.stanford.edu/%7Eknuth/abcde.html"
}
1
\end{verbatim}

In the last reference url field the code \verb+%7E+ will translate into \verb+~+ once clicked in the final pdf.

References are created using command \texttt{\textbackslash cite\{einstein\}}, showing as \citep{einstein}. Other examples: \citep{latexcompanion,knuthwebsite}.

Citation style can be negotiated with the supervisor. See some options in \url{https://www.sharelatex.com/learn/Bibtex_bibliography_styles}.

\section{Crossreferences}

Appendix~\ref{appendix:model} on page~\pageref{appendix:model} contains some additional material.

\chapter{From tex to pdf}

In Linux, run \texttt{pdflatex filename.tex} and \texttt{bibtex filename.tex} repeatedly until no more warnings are shown. This process can be automised using make-command.
 
\chapter{Conclusions\label{chapter:conclusions}}

It is good to conclude with a summary of findings. You can also use separate chapter for discussion and future work. These details you can negotiate with your supervisor.

%%%%%%%%%%%%%%%%%%%%%%%%%%%%%%%%%%%%%%%%%%%%%%%%%%%%%%%%%
\cleardoublepage                          %fixes the position of bibliography in bookmarks
\phantomsection
\addcontentsline{toc}{chapter}{\bibname}  % This lines adds the bibliography to the ToC
\printbibliography

%%%%%%%%%%%%%%%%%%%%%%%%%%%%%%%%%%%%%%%%%%%%%%%%%%%%%%%%%
\backmatter
\begin{appendices}

    %\input{instructions_english}

    \appendix{Sample Appendix\label{appendix:model}}
    usually starts on its own page, with the name and number of the appendix at the top.
    The appendices here are just models of the table of contents and the presentation. Each appendix
    Each appendix is paginated separately.

    In addition to complementing the main document, each appendix is also its own, independent entity.
    This means that an appendix cannot be just an image or a piece of programming, but the appendix must explain its contents and meaning.

\end{appendices}
%%%%%%%%%%%%%%%%%%%%%%%%%%%%%%%%%%%%%%%%%%%%%%%%%%%%%%%%%

\end{document}
